%
% GNU courseware, XIN YUAN, 2018
%

\section{软件框架选择}

\frame{
\centerline{\textbf{\Huge{软件框架选择}}}
}

\frame{\frametitle{分组}
\begin{enumerate}
\item<1-> 前台、后台
\item<2-> 前台、中台、后台
\end{enumerate}
}

\frame{\frametitle{类别}
\begin{itemize}
\item<1-> 数据库系统的框架
\item<2-> 内存数据库/对象数据库的框架
\item<3-> 存储/文件系统的框架
\item<4-> 应用中间件的框架(网络代理,微服务,消息中间件,服务网格,大数据,无服务,低代码)
\item<5-> web服务器的框架
\item<6-> web前端的框架
\end{itemize}
}

\frame{\frametitle{类别}
\begin{itemize}
\item<1-> 测试(前端、后端、中台)框架(功能和非功能)
\item<2-> 功能测试框架和工具(单元测试,集成测试,系统测试,验收测试)
\item<3-> 监控/弹性框架(压力测试,性能测试)
\item<4-> 安全框架(模糊测试)
\end{itemize}
}

\frame{\frametitle{基本思想}
\begin{enumerate}
\item<1-> 解开耦合
\item<2-> 开闭原则
\item<3-> 云原生,固定流程工作往基础设施放
\end{enumerate}
}

\frame{\frametitle{后台web应用}
\begin{enumerate}
\item<1-> 单体/烟囱式,返回html/js页面,前端瘦客户端处理,多页面应用
\item<2-> 微服务架构,返回json数据,前端js富客户端处理,单页面或少数页面的应用
\end{enumerate}
}

\frame{\frametitle{程序框架}
\begin{enumerate}
\item<1-> 后端程序(MVC)
\item<2-> 前端程序(MVC,MVVM)
\end{enumerate}
}

\frame{\frametitle{分工与工具}
\begin{enumerate}
\item<1-> 可分为前端小组、后端小组(测试小组为可选)
\item<2-> 在统一使用协同开发工具的基础上,各小组再使用本组特殊工具
\item<3-> 编写程序代码时,小组成员根据任务的分派完成框架所需的某个类的开发
\end{enumerate}
}

%end
